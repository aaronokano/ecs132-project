\documentclass{article}
\usepackage{amsmath, amsthm, amssymb}
\usepackage{listings}
\usepackage{graphicx}
\usepackage{float}
\usepackage{enumerate}
\usepackage{fancyhdr}
\usepackage[labelfont=bf]{caption}
\usepackage[left=0.75in, top=1in, right=0.75in, bottom=1in]{geometry}

\title{ECS 132 Final Project}  % Declares the document's title.
\author{Aaron Okano, Anatoly Torchinsky, Justin Maple, Samuel Huang }    % Declares the author's name.
\date{November 19, 2012}   % Deleting this command produces today's date.

\begin{document}           % End of preamble and beginning of text.

\maketitle                 % Produces the title.

\section{Forest Fire}

In [Cortez and Morais, 2007], the output 'area' was first transformed with a ln(x+1) function. 
Then, several Data Mining methods were applied. After fitting the models, the outputs were 
post-processed with the inverse of the ln(x+1) transform. Four different input setups were 
used. The experiments were conducted using a 10-fold (cross-validation) x 30 runs. Two 
regression metrics were measured: MAD and RMSE. A Gaussian support vector machine (SVM) fed 
with only 4 direct weather conditions (temp, RH, wind and rain) obtained the best MAD value: 
12.71 +- 0.01 (mean and confidence interval within 95% using a t-student distribution). The 
best RMSE was attained by the naive mean predictor. An analysis to the regression error curve 
(REC) shows that the SVM model predicts more examples within a lower admitted error. In effect, 
the SVM model predicts better small fires, which are the majority. Our goal with this data
is to predict the fire size.

\section{Parkinson's disease}

The dataset was created by Max Little of the University of Oxford, in collaboration with the
National Centre for Voice and Speech, Denver, Colorado, who recorded the speech signals.
The original study published the feature extraction methods for general voice disorders.

This dataset is composed of a range of biomedical voice measurements from 31 people, 23 with
Parkinson's disease (PD). Each column in the table is a particular voice measure, and each row
corresponds one of 195 voice recording from these individuals ("name" column). The main aim of
the data is to discriminate healthy people from those with PD, according to "status" column
which is set to 0 for healthy and 1 for PD.

The data is in ASCII CSV format. The rows of the CSV file contain an instance corresponding to
one voice recording. There are around six recordings per patient, the name of the patient is
identified in the first column.For further information or to pass on comments, please contact
Max Little (littlem '@' robots.ox.ac.uk).

Our goal with this data is to try to monitor patients with Parkinson's disease remotely, by
simply analyzing their voices on the phone.

\section{Beyond ECS 132}

Probability and statistics are not just applicable to this class, but also to other fields of CS.
Here are/is research papers written by some members of our CS faculty, in which they make use
of probability or statistics.

\end{document}             % End of document.